\documentclass[12pt]{article}
\usepackage{algorithm2e}
\usepackage{amsmath}
\usepackage{changepage}
\usepackage{pythonhighlight}
\usepackage{titling}
\usepackage{xcolor}

% algorithm2e
\RestyleAlgo{ruled}
\SetKwProg{Fn}{Function}{}{end}
\SetKw{Stop}{stop}

% titling
\title{\textit{Data Structures \& Algorithms} Cheat Sheet}
\setlength{\droptitle}{-20ex}
\author{Thomas Monson}
\renewcommand\maketitlehookb{\vspace{-3ex}}
\date{}
\renewcommand\maketitlehookd{\vspace{-2ex}}

\setlength\parindent{0pt}

\newcommand{\imply}[1]{
  \-\hspace{1em}$\implies$ \parbox[t]{11.2cm}{#1}
}

\begin{document}
\maketitle

\section*{Essential Patterns}
\hrule\vspace{5ex}

\subsection*{Dynamic Programming}

\begin{adjustwidth}{2em}{0pt}
Optimal substructure $\implies$ divide and conquer \medskip\\
Optimal substructure + greedy choice $\implies$ greedy \medskip\\
Optimal substructure + overlapping subproblems \\
\-\hspace{1em}$\implies$ dynamic programming \\

\textbf{Would it be helpful to rephrase a problem in order to more easily define its subproblems?} \\

\-\hspace{1em}\fbox{\begin{minipage}{28em}
  Given an integer array, return the length of the longest strictly increasing subsequence (LIS). \smallskip\\
  \-\hspace{1em}$\equiv$\hspace{0.5em} Return the length of the LIS of an array \texttt{a} of length $n$. \\
  \-\hspace{1em}$\equiv$\hspace{0.5em} Return the length of the LIS of \texttt{a[0:n]}. \bigskip\\
  The LIS of \texttt{a} must have some first element. If this is the $i$th element, then the LIS of \texttt{a} is equal to the LIS of \texttt{a[i:]}, where \texttt{a[i]} is the first element of the sequence. \medskip\\
  Let \texttt{dp[i]} be the length of the LIS of \texttt{a[i:]}, where \texttt{a[i]} is the first element of the sequence. Return \texttt{max(dp)}.
\end{minipage}} \bigskip

\item \texttt{@functools.lru\_cache}
\end{adjustwidth}

\subsection*{Arrays}

\begin{adjustwidth}{2em}{0pt}
\textbf{Would it help to know the sum of elements for any subarray in O(n) time?} \medskip

Computing the \textbf{prefix sum} of an array \texttt{a} will give you the sum of elements for subarrays \texttt{[a[:i] for i in range(1, len(a))]}. By subtracting elements of the prefix sum from each other, you can get the sum of elements for any subarray. That is, \texttt{sum(a[x:y]) = sum(a[:y]) - sum(a[:x])} for $x < y$. \\

\textbf{Would it help to know if two multisets are permutations of each other?} \medskip

\textit{Fundamental theorem of arithmetic: every integer greater than 1 can be represented uniquely as a product of prime numbers.} \medskip

You can design a hash function that uses \textbf{prime factorization} to map multisets to unique integers. For example, you can map all permutations (anagrams) of a string to a unique integer like so: \medskip

\begin{python}
def compute_hash(s: str) -> int:
    alphabet_primes = [2, 3, 5, 7, 11, 13, 17, 19, 23, 29,
                       31, 37, 41, 43, 47, 53, 59, 61, 67,
                       71, 73, 79, 83, 89, 97, 101]
    h = 1
    for ch in s:
        h *= alphabet_primes[ord(ch) - ord('a')]
    return h
\end{python}


\end{adjustwidth}

\subsection*{Searching}

\begin{adjustwidth}{2em}{0pt}
Here's some code for a binary search:
\begin{python}
def binary_search(nums: list[int], target: int) -> int:
    left, right = 0, len(nums) - 1
    while left <= right:
        mid = (left + right) // 2
        if nums[mid] < target:
            left = mid + 1
        elif nums[mid] > target:
            right = mid - 1
        else:
            return mid
    return -1
\end{python}

\texttt{bisect} (binary search)
\end{adjustwidth}

\subsection*{Sorting}

\begin{adjustwidth}{2em}{0pt}
\textbf{Do you need to sort items according to a custom scheme?}
\begin{itemize}
  \item \texttt{functools.cmp\_to\_key} 
  \item Create class and define dunder methods \texttt{\_\_lt\_\_}, \texttt{\_\_gt\_\_}, \texttt{\_\_le\_\_}, \texttt{\_\_ge\_\_}, \texttt{\_\_eq\_\_}, \texttt{\_\_ne\_\_}
\end{itemize}
\end{adjustwidth}
\bigskip

\begin{adjustwidth}{2em}{0pt}
    \textbf{Do you need to schedule tasks based on their dependencies?} \medskip\\
    You can apply \textbf{topological sorting} to a directed graph. This will produce a linear ordering of the vertices such that for every directed edge \textit{uv} from vertex \textit{u} to vertex \textit{v}, \textit{u} comes before \textit{v}. However, if the graph has cycles, such an ordering does not exist.\medskip\\
    There are two main topological sorting algorithms: \textit{Kahn's algorithm} (BFS) and \textit{cycle detection via DFS}. The former cannot visit cycles and detects them by checking for unvisited nodes after traversal. The latter detects cycles by entering the first one it finds and completing a loop.\\

    \begin{algorithm}[H]
      \SetAlgoLined
      \DontPrintSemicolon
      \KwData{$G = (V, E)$}
      \KwResult{$L$ (list of $v \in V$ in topological order)}
      $L \longleftarrow$\hspace{0.5mm}\texttt{[]}\;
      $S \longleftarrow \{v \in V \mid v \text{ has no incoming edges}\}$\;
      \While{S is not empty}{
        remove a node $n$ from $S$\;
        append $n$ to $L$\;
        \ForEach{node $m$ with an edge $e$ from $n$ to $m$}{
          remove $e$ from $E$\;
          \If{$m$ has no incoming edges}{
            add $m$ to $S$\;
          }
        }
      }
      \eIf{$E$ is empty}{
        \KwRet{$L$}
      }{
        \KwRet{error}\tcc*[r]{the graph has a cycle}
      }
      \caption{Kahn's Algorithm\hspace{13mm}\texttt{/* see A.1 for code */}}
    \end{algorithm}
    \bigskip

    \begin{algorithm}[H]
      \SetAlgoLined
      \DontPrintSemicolon
      \SetKwFunction{FnVisit}{visit}
      \KwData{$G = (V, E)$}
      \KwResult{$L$ (list of $v \in V$ in topological order)}
      $L \longleftarrow$\hspace{0.5mm}\texttt{[]}\;
      \Fn{\FnVisit{node $n$}}{
        \If{$n$ has a permanent mark}{
          \KwRet{}
        }
        \If{$n$ has a temporary mark}{
          \Stop\tcc*[r]{the graph has a cycle}
        }
        mark $n$ with a temporary mark\;
        \ForEach{node $m$ with an edge from $n$ to $m$}{
          visit($m$)\;
        }
        remove temporary mark from $n$\;
        mark $n$ with a permanent mark\;
        prepend $n$ to $L$\;
      }
      \While{$\exists$ nodes without a permanent mark}{
        select an unmarked node $n$\;
        visit($n$)\;
      }
      \KwRet{$L$}\;
      \caption{DFS Topological Sort\hspace{8mm}\texttt{/* see A.2 for code */}}
    \end{algorithm}
\end{adjustwidth}

\subsection*{Other Stuff}

\begin{itemize}
  \item Helper method recursion (parameter or nonlocal)
  \item Kadane's algorithm (maximum subarray)
  \item Knapsack problem (combinatorial optimization)
  \item Sweep line algorithm (convex hull)
  \item Backtracking (DFS)
  \item Sliding window
  \item LRU Cache (hash map + DLL, OrderedDict)
  \item Monotonic stack
  \item Union-find
\end{itemize}

\section*{Useful Python Constructs}
\hrule\vspace{5ex}

Do you need to...
\begin{itemize}
  \item Count items in a collection? \smallskip\\
    \imply{\texttt{collections.Counter} creates a dictionary of the form \texttt{\{element: count\}}}
  \item Return a default value for keys not found in a dictionary? \smallskip\\
    \imply{\texttt{collections.defaultdict}}
  \item Get the ASCII value of a character? \smallskip\\
    \imply{\texttt{ord(ch)}}
  \item Reverse a list? \smallskip\\
    \imply{The fastest method is the ``Martian smiley" \texttt{[::-1]}}
\end{itemize}

\texttt{itertools.combinations}, \texttt{itertools.permutations} \\
\texttt{re} (regex) \\
\texttt{enumerate} $\rightarrow$ \texttt{count, value} \\
\texttt{map}, \texttt{filter}, \texttt{reduce}, \texttt{zip}

\section*{Other}
\hrule\vspace{5ex}

\begin{itemize}
  \item DFS $\rightarrow$ stack (recursion) $\rightarrow$ LIFO
  \item BFS $\rightarrow$ queue (iteration) $\rightarrow$ FIFO
  \item Online tests: have a Python scratchpad open, spam the ``Run Tests" button (EAFP > LBYL)
  \item Number of subarrays of array of size $n$: $\frac{n(n+1)}{2}$
  \item Python is pass-by-assignment
  \begin{itemize}
    \item Immutable objects are pass-by-value
    \item Mutable objects are pass-by-reference
    \item You can rebind the variable in the inner scope, but the outer scope will remain unchanged
  \end{itemize}
\end{itemize}

\section*{Potentially Useful Algorithms}
\hrule\vspace{5ex}

\begin{itemize}
  \item Rabin-Karp (string-searching, uses a rolling hash to make approximate comparisons between substring hash and target hash, makes exact comparison if hashes match)
  \item Kruskal's algorithm and Prim's algorithm (minimum spanning tree)
\end{itemize}

\appendix

\section{Python Code Samples}

\subsection{Topological Sorting - Kahn's Algorithm}
\begin{python}
def find_order_bfs(adj_list: list[list[int]],
                   in_degrees: list[int]) -> list[int]:
    # 1. Create list of start nodes
    queue = deque()
    for n, d in enumerate(in_degrees):
        if d == 0:
            queue.append(n)

    topo_order = []
    while queue:
        # 2. Add a start node n to the topological ordering
        n = queue.popleft()
        topo_order.append(n)

        # 3. Remove edges from n to its neighbors
        #    Add neighbors of in-degree 0 to start node list
        for m in adj_list[n]:
            in_degrees[m] -= 1
            if in_degrees[m] == 0:
                queue.append(m)

    return topo_order if len(topo_order) == len(adj_list) else []
\end{python}

\subsection{Topological Sorting - DFS Cycle Detection}
\begin{python}
def find_order_dfs(adj_list: list[list[int]]) -> list[int]:
    visited = set()
    dfs_tree = set()
    topo_order = []

    def has_cycle(n):
        if n in visited:  # path already explored
            return False
        if n in dfs_tree:  # cycle detected
            return True

        dfs_tree.add(n)
        for m in adj_list[n]:
            if has_cycle(m):
                return True

        dfs_tree.remove(n)
        visited.add(n)
        topo_order.append(n)
        return False

    for n in range(len(adj_list)):
        if has_cycle(n):
            return []
    return topo_order
\end{python}
\end{document}

% \begin{tikzpicture}
% \matrix (m) [matrix of nodes,
%              nodes={draw, minimum size=8mm, anchor=center},
%              nodes in empty cells, minimum height=1cm,
%              row 1/.style={nodes={draw=none}},]
% {
%   \texttt{0} & \texttt{1} & \texttt{2} & \texttt{3} & \texttt{4} & \texttt{5} & \texttt{6} & \texttt{7} & \texttt{8} \\
%   2 & 4 & 4 & 5 &   &   &   &  &\\
% };
% \draw[semithick] (-3.66, -0.9) -- ++(0, -0.1) -- ++(0.81, 0) -- ++(0, 0.1);
% \draw[semithick] (-3.66, -1.1) -- ++(0, -0.1) -- ++(1.62, 0) -- ++(0, 0.1);
% \draw[semithick] (-3.66, -1.3) -- ++(0, -0.1) -- ++(2.43, 0) -- ++(0, 0.1);
% \draw[semithick] (-3.66, -1.5) -- ++(0, -0.1) -- ++(3.24, 0) -- ++(0, 0.1);
% \end{tikzpicture}

